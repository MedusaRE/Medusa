\documentclass{article}

\title{The Medusa Project}
\author{spv}

\begin{document}
	\maketitle

	\section{Notes}
	Note: this document is written from the perspective of the Medusa Project's
	main developer, spv.

	\section{Introduction}
	\textit{Medusa} is a project to create a cross platform, free, and general
	purpose tool for software and hardware research, like reverse engineering,
	analysis, emulation, development, debugging, and other similar tasks. I
	started the \textit{Medusa} project originally under the name of
	\textit{xp$^{DBG}$}, as a hobby project to learn about reverse engineering,
	and because I felt that current development and reverse engineering tools
	all have their own problems, which I wanted to solve.

	\subsection{Current Tools' Issues}
	This subsection is partially paraphrased from the \textit{Medusa} Project's
	\texttt{README.md} file.

	\begin{itemize}
		\item \textit{Cutter}: not very featureful, essentially a
		\textit{radare2} GUI, and doesn't have debugger and/or emulation support
		to my knowledge.
		\item \textit{Ghidra}: personal favorite currently, still doesn't have
		emulation support or code editing, and is written in Java (besides the
		decompiler), which is one of my least favorite languages*.
		\item \textit{IDA Pro}: expensive, non-free, does not have emulation
		support, or code editing.
		\item \textit{Radare2}: does not have code editing, a GUI, or the level
		of emulation support which I intend to include in \textit{Medusa}.
		\item \textit{Binary Ninja}: I honestly do not have a lot of experience
		with \textit{Binary Ninja}, but to my knowledge, it is not free / 
		open-source software, it isn't a full IDE (like \textit{Medusa} is
		intended to be), and doesn't have emulation support (like
		\textit{Medusa} is intended to).
	\end{itemize}

	* Since the writing of \texttt{README.md}, I have changed my opinion slightly
	in regards to Java. I still dislike parts of the language (particularly,
	requiring a VM), but I can appreciate other parts of it.

	\subsection{\textit{Medusa}'s Solutions}
	For a first example, take \textit{Unicorn}. \textit{Unicorn} is a
	library/API based on \textit{QEMU}, that provides an interface to control
	virtualized/emulated CPUs, and general machines. I do appreciate the
	\textit{Unicorn} project's work, but I think it has some flaws. (or rather,
	it is not the perfect library for the \textit{Medusa} Project's goals.)

	To solve some of \textit{Unicorn}'s issues, the Medusa Project has a
	subproject / subcomponent called \textit{libmedusa}. \textit{libmedusa} is
	a C++ library with a "standardized" API for interfacing with emulated
	machines ("soft silicon", as I call it), as well as real machines ("hard
	silicon", as I call it). \textit{libmedusa} also provides a "standardized"
	API for interacting with other types of components, such as displays, sound
	outputs, and other components useful when controlling, say, an emulated
	Commodore 64. If you wanted to do so, you could provide an implementation of
	the \textit{libmedusa} API for emulators for the 6502, SID, VIC-II, and
	other components. (or even real hardware!) Then, other software can
	interface with an emulated, or even a real C64, without needing to be
	specifically written to support it.

	Another way that this API could be useful is if you (or your company) is
	developing a new piece of hardware. \textit{Medusa} (or other software)
	probably doesn't support unreleased hardware, and with other software, say
	\textit{IDA Pro} (or something with emulation support) it may be difficult
	to emulate your hardware elegantly for testing purposes. With
	\textit{libmedusa}, you could implement its API for your particular display,
	sound output, CPU, etc, (or even re-use existing implementations, if, say,
	you use a standard CPU ISA, like \textit{ARMv8}), and software can interact
	with your hardware without needing to be specifically written to support it.

	\textit{libmedusa} doesn't just provide an alternative to \textit{Unicorn}.
	It provides an all-in-one API that can replace \textit{Unicorn},
	\textit{Capstone}, \textit{Keystone}, \textit{LIEF} (\textit{libmedusa}
	provides an API for parsing formats like ELF), and other libraries.
\end{document}