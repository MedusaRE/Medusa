\documentclass{report}

\usepackage[pdftex, pdfborderstyle={/S/U/W 0}, breaklinks=false]{hyperref}
\usepackage{background}
\usepackage{graphicx}
\usepackage{listings}
\usepackage{comment}
\usepackage{ccicons}
\usepackage{xspace}

%\usepackage[margin=1.8in,footskip=0.25in]{geometry}

\newcommand{\xpDBG}{\textit{xp$^{DBG}$}\xspace}
\newcommand{\Medusa}{\textit{Medusa}\xspace}

\lstset{
	basicstyle=\ttfamily,
	columns=fullflexible,
	keepspaces=true,
	tabsize=4
}

\begin{comment}
	Add your preferred name to the \author section if you edit this file.
\end{comment}

\title{The Medusa Project}
\author{spv}

\SetBgContents{\textnormal{This file is licensed under the terms of the CC BY-SA
						   4.0 license. \ccbysa}}
\SetBgScale{1}
\SetBgAngle{0}
\SetBgOpacity{1}
\SetBgColor{black}
\SetBgPosition{current page.south}
\SetBgVshift{1cm}

\begin{document}
	\maketitle

	\chapter{Notes}
	\begin{enumerate}
		\item This document / paper / book is mainly written from the
		perspective of the \textit{Medusa} Project's main developer, spv.

		\item All software released by the \textit{Medusa} Project is, whenever
		possible, released under the GNU General Public License, Version 2
		(without the "any later version" clause).

		\item This document / paper / book was partially written by me (spv) in
		order to learn \LaTeX.

		\item Documentation for \textit{Medusa} Project API's can be found at
		
		\href{https://docs.medusa-re.org}{docs.medusa-re.org}.

		\item A wiki for the \textit{Medusa} Project can be found at
		\href{https://wiki.medusa-re.org}{wiki.medusa-re.org}.

		\item When a first-person pronoun is used, and then a name appears in
		(parentheses), that first-person pronoun refers to the name in
		parentheses. For example: My (spv) cat is cute.
	\end{enumerate}

	\chapter{Introduction}
	\textit{Medusa} is a project to create a cross platform, free (libre), and
	general purpose tool for software and hardware research, like reverse
	engineering, analysis, emulation, development, debugging, and other similar
	tasks. I started the \textit{Medusa} project originally under the name of
	\xpDBG, as a hobby project to learn about reverse engineering, and because I
	felt that current development and reverse engineering tools all have their
	own problems, which I wanted to solve.

	\section{Current Tools' Issues}
	This section is partially paraphrased from the \textit{Medusa} Project's
	\texttt{README.md} file.

	\begin{itemize}
		\item \textit{Cutter}: not very featureful, essentially a
		\textit{radare2} GUI, and doesn't have debugger and/or emulation support
		to my knowledge.
		\item \textit{Ghidra}: personal favorite currently, still doesn't have
		emulation support or code editing, and is written in \textit{Java}
		(besides the decompiler), which is one of my least favorite languages*.
		\item \textit{IDA Pro}: expensive, non-free, does not have emulation
		support, or code editing.
		\item \textit{Radare2}: does not have code editing, a GUI, or the level
		of emulation support which I intend to include in \textit{Medusa}.
		\item \textit{Binary Ninja}: I honestly do not have a lot of experience
		with \textit{Binary Ninja}, but to my knowledge, it is not free / 
		open-source software, it isn't a full IDE (like \textit{Medusa} is
		intended to be), and doesn't have emulation support (like
		\textit{Medusa} is intended to).
	\end{itemize}

	* Since the writing of \texttt{README.md}, I have changed my opinion
	slightly in regards to Java. I still dislike parts of the language
	(particularly, requiring a VM), but I can appreciate other parts of it.

	\section{\textit{Medusa}'s Solutions}
	\subsection{\textit{libmedusa}} \label{solution_libmedusa1}
	For a first example, take \textit{Unicorn}. \textit{Unicorn} is a
	library/API based on \textit{QEMU}, that provides an interface to control
	virtualized/emulated CPUs, and general machines. I do appreciate the
	\textit{Unicorn} project's work, but I think it has some flaws. (or rather,
	it is not the perfect library for the \textit{Medusa} Project's goals.)

	To solve some of \textit{Unicorn}'s issues, the \textit{Medusa} Project has
	a subproject / subcomponent called \textit{libmedusa}. \textit{libmedusa} is
	a C++ library with a "standardized" API for interfacing with emulated
	machines ("soft silicon", as I call it), as well as real machines ("hard
	silicon", as I call it). \textit{libmedusa} also provides a "standardized"
	API for interacting with other types of components, such as displays, sound
	outputs, and other components useful when controlling, say, an emulated
	Commodore 64. If you wanted to do so, you could provide an implementation of
	the \textit{libmedusa} API for emulators for the 6502, SID, VIC-II, and
	other components. (or even real hardware!) Then, other software can
	interface with an emulated, or even a real C64, without needing to be
	specifically written to support it.

	Another way that this API could be useful is if you (or your company) is
	developing a new piece of hardware. \textit{Medusa} (or other software)
	probably doesn't support unreleased hardware, and with other software, say
	\textit{IDA Pro} (or something with emulation support) it may be difficult
	to emulate your hardware elegantly for testing purposes. With
	\textit{libmedusa}, you could implement its API for your particular display,
	sound output, CPU, etc, (or even re-use existing implementations, if, say,
	you use a standard CPU ISA, like \textit{ARMv8}), and software can interact
	with your hardware without needing to be specifically written to support it.

	\textit{libmedusa} doesn't just provide an alternative to \textit{Unicorn}.
	It provides an all-in-one API that can replace \textit{Unicorn},
	\textit{Capstone}, \textit{Keystone}, \textit{LIEF} (\textit{libmedusa}
	provides an API for parsing formats like ELF), and other libraries.

	\subsection{\textit{barcelona}}
	This subsubsection is adapted from the \textit{barcelona} subcomponent /
	subproject's \texttt{README.md} file.

	\textit{barcelona} is a subproject of the \textit{Medusa} Project to create
	a TUI (Terminal User Interface)-based IDE (Integrated Development
	Environment).

	\subsection{\textit{paris}}
	This subsubsection is adapted from the \textit{paris} subproject's
	\texttt{README.md}.

	\textit{paris} is a subcomponent of the \textit{Medusa} Project to develop
	the client/server architecture that is intended to be used in the project.
	\textit{Medusa} is meant to be modeled after a client/server architecture,
	where a machine (or machines) operates the server, and handle the bulk of
	the processing work; and a machine (or machines) runs a client, which
	connects to the server, and provides a UI to interface with the server.

	The server can be the same machine as the client, and it does not need to be
	over the network (i.e. TCP), it could be a socket, for example.

	\subsection{\textit{rome}}
	This subsubsection is adapted from the \textit{rome} subcomponent's
	\texttt{README.md} file.

	Rome is a subcomponent of the \textit{Medusa} Project to write a modern C++
	TUI framework based on \textit{ncurses}. \textit{ncurses} / \textit{curses}
	is a great framework, but in my (spv) opinion, it feels a bit ancient
	compared to what could be done with, say, C++. It's a bit low-level, and the
	code you write with it is, in my opinion, not the best-looking, to put it
	mildly.

	I do understand the historical reason(s) for the \textit{curses} API
	essentially using $(y,x)$ instead of $(x,y)$, but it doesn't make it any
	less strange to write.

	\chapter{History}
	\section{p0laris}
	The \textit{Medusa} Project's history begins with another one of my
	projects, \textit{p0laris}. I, at the time, was writing shellcode to execute
	from within kernel-land on iOS 9, and wanted to debug said shellcode. I did
	not (and still don't, at the time of writing) own a DCSD-cable, or other
	similar means to debug the shellcode, so I set out to develop a debugger
	that I could run on my computer.

	\section{ARMistice}
	I discussed \xpDBG in the \textit{Introduction} section, but before even
	\xpDBG, I created a small project called \textit{ARMistice}.
	\textit{ARMistice} was a small \textit{Python} script that emulated an
	\textit{ARMv7} CPU using \textit{Unicorn}, and provided memory editing,
	assembly, disassembly, and a UI with \textit{ncurses}, \textit{Keystone},
	and \textit{Capstone}. This little \textit{Python} script was the genesis
	for what would eventually become the \textit{Medusa} Project.

	\section{\xpDBG}
	After \textit{ARMistice}, as was stated in the \textit{Introduction}
	section, the \textit{Medusa} project was started under a different name,
	\xpDBG. \xpDBG stands (or rather, stood) for "cross (X) Platform
	DeBuGger", which references the project's goal to create a cross platform
	suite for software and hardware research. 

	\section{Name change}
	I changed the name from \xpDBG to \Medusa mainly for branding reasons. I
	thought of the name (I forget how), and I thought it sounded cool. I also
	figured that it was early enough in the project's life that it wouldn't be
	too damaging to do so. I still own the domains
	\href{https://xpdbg.com}{xpdbg.com} and \href{https://xpdbg.org}{xpdbg.org},
	but they (at the time of writing, at least) just contain partially broken
	versions of the \Medusa website.

	\chapter{Documentation}
	\section{\textit{libmedusa}}
	\

	Note: see \ref{solution_libmedusa1} for an introduction to this subsection.

	Note: this code is subject to change.

	\

	\subsection{\textit{Machine}}

	This subsection will document code examples for \textit{libmedusa}, a
	subcomponent / subproject of the \Medusa Project. Let's use, for example
	\texttt{ARMv7Machine}. \texttt{ARMv7Machine} is a class that implements the
	generic \texttt{Machine} class, which is a "standardized" interface between
	software utilizing \textit{libmedusa}, and soft-or-hard silicon machines
	(specifically, CPUs).
	\

	\

	Requirements:
	\begin{itemize}
		\item A computer.

		\item \textit{libmedusa} installed on said computer. (Exact directions
		may differ from OS to OS, but in general, clone the \Medusa Project's
		\textit{git} repository, cd into
		\texttt{Medusa/medusa\_software/libmedusa}, and run \texttt{make},
		followed by \texttt{sudo make install}. This will require
		\textit{Unicorn}, \textit{Keystone}, \textit{Capstone}, and possibly
		other components installed. Check \texttt{README.md} for a full list of
		requirements.)

		\item Your preferred IDE for editing C++ code. (in the future, Medusa
		will hopefully be able to be used for this purpose. (specifically, the
		london subproject / subcomponent which will be integrated into the
		larger tool. ))

		\item A compatible C++ compiler: we use \textit{LLVM / Clang}.
		\footnote{
			\label{llvm_clang1}
			\href{https://www.llvm.org}{llvm.org}
		}.
	\end{itemize}

	Create a new source file, and \texttt{\#include}
	\texttt{<libmedusa/libmedusa.hpp>}, \

	\texttt{<libmedusea/Machine.hpp>}, and \texttt{<libmedusa/ARMv7Machine.hpp>}.
	\
	
	\

	To create a new \texttt{ARMv7Machine}, declare it like you would any other
	instance of a class.

	\begin{lstlisting}
libmedusa::ARMv7Machine armv7_machine;
	\end{lstlisting}

	Now, let's map some memory to place the code we'd like to emulate in.

	\begin{lstlisting}
/*
 *  a mem_reg_t is a libmedusa type representing a memory region, containing
 *  information about its address, size, and memory protections.
 */
libmedusa::mem_reg_t region;

/*
 *  set region.addr to 0x0, size to 0x10000, and allow RWX in the memory.
 *  this describes a memory region at the beginning of the CPU's memory
 *  space, that is 0x10000 bytes long, and allows all operations, read,
 *  write, and execute.
 */
region.addr = 0x0;
region.size = 0x10000;
region.prot = XP_PROT_READ | XP_PROT_WRITE | XP_PROT_EXEC;

/*
 *  Machine::map_memory(mem_reg_t region) will map the region into memory.
 */
armv7_machine.map_memory(region);
\end{lstlisting}

	Now, let's copy some \textit{ARMv7} code into memory to be executed.

	First, we need to define an array containing our code.

	\begin{lstlisting}
uint8_t test_arm_thumb_code[] = {
	0x41, 0x20,				//	movs	r0,	#0x41
	0x40, 0xF2, 0x20, 0x40,	//	movw	r0,	#0x420
	0x40, 0xF2, 0x69, 0x01,	//	movw	r1,	#0x69
	0xA0, 0xEB, 0x01, 0x00,	//	sub		r0,	r0,	r1
	0x01, 0x44,				//	add		r1,	r1,	r0
	0x00, 0x00,				//  mov		r0,	r0
};
	\end{lstlisting}

	The \textit{libmedusa} API uses \texttt{std::vector}'s for a lot of work
	that would normally be done by standard arrays or pointers. So, we must
	create an \texttt{std::vector} containing our \textit{ARMv7} code.

	\begin{lstlisting}
vector<uint8_t> tmp_vector(test_arm_thumb_code,
						   test_arm_thumb_code
						 + sizeof(test_arm_thumb_code));
	\end{lstlisting}

	Now, let's copy our code \texttt{vector} into the \texttt{ARMv7Machine}'s
	memory space.

	\begin{lstlisting}
armv7_machine.write_memory(0, tmp_vector);
	\end{lstlisting}

	Next, let's set the \texttt{pc} register's value to \texttt{0x1}. Remember,
	in 32-bit \textit{ARM}, having the LSB set in the \texttt{pc} indicates
	executing \textit{THUMB} code, rather than \textit{ARM} code.

	\begin{lstlisting}
/*
 *  This API should be improved. For now, you have to include the register name
 *  and description, but in the future, that will probably not be necessary.
 *  I'll probably write another version of the `set_register' and `get_register`
 *  functions that take just a register name or id later.
 *
 *  reg_t is another libmedusa-specific type. It contains fields describing a
 *  processor register, with fields such as the name, a short description of the
 *  register's function, an ID specific to this register (unique per CPU, so 2
 *  different CPU's can share an ID, as long as it is not shared within the same
 *  CPU), and the register's value.
 */
libmedusa::reg_t reg;

reg.reg_description = "pc";
reg.reg_name = "pc";
reg.reg_id = 0xf;
reg.reg_value = 0x1;

/*
 *  Machine::set_register(reg_t register) sets the value of a register.
 */
armv7_machine.set_register(reg);
	\end{lstlisting}

	Now, let's step through a few instructions, and print the state of all
	registers after each step.

	\begin{lstlisting}
const int number_of_registers_to_step_through = 0x8;
for (int i = 0;
		 i < number_of_registers_to_step_through;
		 i++) {
	/*
	 *  Machine::get_registers() returns an `std::vector` of all of the CPU's
	 *  registers, as `reg_t`'s.
	 */
	vector<libmedusa::reg_t> registers = armv7_machine.get_registers();
	for (libmedusa::reg_t& i : registers) {
		printf("-----\n"
			   "%s %s %lx %lx\n",
			   i.reg_description.c_str(),
			   i.reg_name.c_str(),
			   i.reg_id,
			   i.reg_value);
	}
	
	/*
	 *  step_instruction is a pre-processor #define for exec_code_step.
	 *  Machine::exec_code_step() steps forward one instruction on the CPU.
	 */
	armv7_machine.step_instruction();
}
	\end{lstlisting}

	Congratulations! You just wrote your first program with \textit{libmedusa}.
	\
	
	\

	\

	\subsection{\textit{libmedusa} ~ \texttt{Component}s}
	\textit{libmedusa}'s emulation / control framework is based on
	\texttt{Component}s. A \texttt{Component} is a generic class that describes
	a "thing" that can be controlled, can give output, or both.

	For example, a \texttt{DisplayOutput} \texttt{Component} describes a display
	output: a framebuffer / bitmap. A \texttt{SoundOutput} \texttt{Component}
	could provide an interface to a \textit{DAC}, or a \texttt{CPU}
	\texttt{Component} could provide a generic interface to a \textit{CPU}, that
	could be implemented by a class to interact with, say, an \textit{ARMv7} or
	\textit{x86(\_64)} processor.

	Let's implement an example \texttt{DisplayOutput} \texttt{Component}.
	\footnote{
		\label{ExampleDisplayComponent1}
		This essentially serves as documentation for both the
		\texttt{DisplayOutput} generic class, and the
		\texttt{ExampleDisplayComponent} implementation of said generic class.

		Most code is stolen/paraphrased from the
		\texttt{ExampleDisplayComponent} implementation.
	}

	First, we need to install \textit{libmedusa}. See the previous subsection
	for instructions.

	Create a class implementing the \texttt{DisplayOutput} generic class in a
	header file.

	\begin{lstlisting}
class ExampleDisplayComponent : public DisplayOutput {
	...
}
	\end{lstlisting}

	Next, \texttt{\#include} \texttt{<libmedusa/libmedusa.hpp>},
	\texttt{<libmedusa/Component.hpp>}, and the header file that contains your
	implementation.

	\section{\textit{rome}}
	As stated previously, \textit{rome} is a subcomponent / subproject of the
	\Medusa Project to create a C++ library / API, as a sort-of replacement for
	\textit{ncurses}. I thought that \textit{ncurses}'s API felt quite dated, so
	I started the \textit{rome} subproject.

	Let's write some code with \textit{rome}.

	Requirements:
	\begin{itemize}
		\item A computer.
		
		\item \textit{rome} "installed". Currently, \textit{rome} does not have
		an install rule in its Makefile, but in the future, \texttt{git clone}
		the \Medusa Project \textit{git} repository as usual, then \texttt{cd
		Medusa/medusa\_software/rome}, and \texttt{make}.

		\item Your preferred IDE for editing C++ code. (in the future, Medusa
		will hopefully be able to be used for this purpose. (specifically, the
		london subproject / subcomponent which will be integrated into the
		larger tool. ))

		\item A compatible C++ compiler: we use \textit{LLVM / Clang}.
		\footnote{
			See footnote \ref{llvm_clang1}.
		}
	\end{itemize}

	First, \texttt{\#include <rome/rome.hpp>}. Now, create a
	\texttt{rome::window}.

	\begin{lstlisting}
rome::window window;
	\end{lstlisting}

	Now, let's add some text!

	\begin{lstlisting}
/*
 *  rome::window::addstr(std::string str, int x, int y) prints a
 *  string `str` to the screen at position `(x, y)`.
 */
window.addstr("This is rome example code.", 3, 1);
	\end{lstlisting}

	How about we reverse that text, like a titlebar? (Background becomes
	foreground, and vice versa. )

	\begin{lstlisting}
/*
 *  rome::window::chgattr(int attr,
 *  					  int x,
 *  					  int y,
 *  					  int n,
 *  					  int color_pair) will set the terminal attributes
 *  `attr` at the location `(x,y)`, for `n` characters, using the `curses` color
 *  pair `color_pair`. Using a negative number for `n` will change the
 *  attributes up until `abs(n)` characters from the end of the line.
 */
window.chgattr(A_REVERSE, 1, 1, -2, 0);
	\end{lstlisting}

	Finally, let's wait for a key, and exit the program.
	\begin{lstlisting}
/*
 *  rome::window::getch() returns an `ncurses` `int` key. We discard the actual
 *  key returned, we just want to wait for a key press.
 */
window.getch();
	\end{lstlisting}

	When the \texttt{rome::window} object is destroyed, the proper / normal
	terminal state is automatically reset. So, we're done!

	Congratulations! You just wrote your first program with \textit{rome}.

	\chapter{Links}
	This chapter contains useful links for the \Medusa Project.

	\begin{enumerate}
		\item \href{https://www.medusa-re.org}{medusa-re.org} ~ the \Medusa
		Project's website

		\item \href{https://docs.medusa-re.org}{docs.medusa-re.org} ~
		documentation for the \Medusa Project's software, libraries, and API's.
		(mostly\footnote{Read: "currently all".} doxygen)

		\item \href{https://wiki.medusa-re.org}{wiki.medusa-re.org} ~ the
		\Medusa Project's wiki

		\item \href{https://www.gitlab.com/MedusaRE}{gitlab.com/MedusaRE}
		~ the \textit{GitLab} organization / group for the \Medusa Project

		\item \href{https://www.gitlab.com/MedusaRE/Medusa}
				   {gitlab.com/MedusaRE/Medusa}
		~ the \textit{GitLab} repository for the \Medusa Project

		\item \href{https://www.gitlab.com/MedusaRE/wiki}
				   {gitlab.com/MedusaRE/wiki}
		~ the \textit{GitLab} repository for the \Medusa Project's wiki

		\item \href{https://www.gitlab.com/MedusaRE/Medusa.git}
				   {https://gitlab.com/MedusaRE/Medusa.git}
		~ the \textit{git} repository for the \Medusa Project, hosted by
		\textit{GitLab}

		\item \href{https://www.gitlab.com/MedusaRE/wiki.git}
				   {https://gitlab.com/MedusaRE/wiki.git}
		~ the \textit{git} repository for the \Medusa Project's wiki, hosted by
		\textit{GitLab}

		\item \href{https://git.medusa-re.org/Medusa.git}
				   {https://git.medusa-re.org/Medusa.git}
		~ a "futureproof" \textit{git} repository URL for the \Medusa Project
		\footnote{
			\label{git_note}
			While the \href{https://git.medusa-re.org}{git.medusa-re.org} domain
			is hosted by the \Medusa Project (or even more specifically, my
			home server), the \textit{git} repositories you can access from it
			(namely, \texttt{Medusa.git} and \texttt{wiki.git}) are not. They
			are currently hosted by \textit{GitLab}. I simply set up my
			\textit{Apache} installation to redirect
			\textit{git.medusa-re.org/*} to
			\textit{https://gitlab.com/MedusaRE/*}. This is done so that, in the
			unlikely case that the \Medusa Project has to move away from
			\textit{GitLab}, \textit{git} repository URL's (and remote URL's
			used by \textit{git} for pushes and such) will not have to be
			updated.
		}

		\item \href{https://www.gitlab.com/MedusaRE/wiki.git}
				   {https://git.medusa-re.org/wiki.git}
		~ a "futureproof" git repository URL for the \Medusa Project's wiki
		\footnote{See footnote \ref{git_note}.}
	\end{enumerate}

	\chapter{Contributors}
	The \Medusa Project is a pet-project of mine. As such, most
	contributions have been made by me. All contributors (including me) are
	listed below.

	\begin{itemize}
		\item spv (\textit{\href{mailto:spv@spv.sh}{spv@spv.sh}})
	\end{itemize}

	\chapter{Credits}
	The \Medusa Project might not have been possible without the work of many
	others. A non-exhaustive list of individuals/entities that deserve credit
	for their work is provided below. Note that none of these entities are
	necessarily associated with the \Medusa Project, just that their work helped
	the project.

	\begin{itemize}
		\item spv (\textit{\href{mailto:spv@spv.sh}{spv@spv.sh}}) ~ founding the
		project

		\item Nguyen Anh Quynh
		(\textit{\href{mailto:aquynh@gmail.com}{aquynh@gmail.com}}) ~
		\textit{Capstone}, \textit{Keystone}, \textit{Unicorn}

		\item The \textit{QEMU} Team (\href{https://www.qemu.org}{qemu.org}) ~
		\textit{QEMU}

		\item \textit{LIEF} Project
		(\href{https://lief-project.github.io/}{lief-project.github.io}) ~
		\textit{LIEF}

		\item The \textit{GNOME} Project
		(\href{https://www.gnome.org}{gnome.org}) ~ \textit{GTK}, \textit{GTKMM}
	\end{itemize}
\end{document}
