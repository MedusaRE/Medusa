\documentclass{article}

\title{The Medusa Project}
\author{spv}

\begin{document}
	\maketitle

	\section{Notes}
	Note: this document is written from the perspective of the Medusa Project's
	main developer, spv.

	\section{Introduction}
	\textit{Medusa} is a project to create a cross platform, free, and general
	purpose tool for software and hardware research, like reverse engineering,
	analysis, emulation, development, debugging, and other similar tasks. I
	started the \textit{Medusa} project originally under the name of
	\textit{xp$^{DBG}$}, as a hobby project to learn about reverse engineering,
	and because I felt that current development and reverse engineering tools
	all have their own problems, which I wanted to solve.

	\subsection{Current Tools' Issues}
	This subsection is partially paraphrased from the \textit{Medusa} Project's
	\texttt{README.md} file.

	\begin{itemize}
		\item Cutter: not very featureful, essentially a radare2 GUI, and
		doesn't have debugger and/or emulation support to my knowledge.
		\item Ghidra: personal favorite currently, still doesn't have emulation
		support or code editing, and is written in Java (besides the
		decompiler), which is one of my least favorite languages*.
		\item IDA (Pro): expensive, non-free, does not have emulation support,
		or code editing.
		\item Radare2: does not have code editing, a GUI, or the level of
		emulation support which I intend to include in Medusa.
		\item Binary Ninja: I honestly do not have a lot of experience with
		Binary Ninja, but to my knowledge, it is not free/open-source software,
		it isn't a full IDE (like Medusa is intended to be), and doesn't have
		emulation support (like Medusa is intended to).
	\end{itemize}

	* Since the writing of \texttt{README.md}, I have changed my opinion slightly
	in regards to Java. I still dislike parts of the language (particularly,
	requiring a VM), but I can appreciate other parts of it.

	\subsection{\textit{Medusa}'s Solutions}
	For a first example, take \textit{Unicorn}. \textit{Unicorn} is a
	library/API based on \textit{QEMU}, that provides an interface to control
	virtualized/emulated CPUs, and general machines. I do appreciate the
	\textit{Unicorn} project's work, but I think it has some flaws. (or rather,
	it is not the perfect library for the \textit{Medusa} Project's goals.)
\end{document}